\documentclass[a4paper]{ctexart}
\usepackage{lmodern}
\usepackage{amsmath}
\usepackage{amsthm}
\usepackage[textwidth=14.5cm]{geometry}
\usepackage{blindtext}
\parindent=0pt

\title{算法分析与设计 - HW1}
\author{}
\date{2022/09/08}

\begin{document}
\begin{sloppypar}  % 控制页边距,防止行溢出

    % Title:
    \maketitle

    % Section 1:
    \section{一些定理的证明}
    \vspace{1em}  % 空一行

    % Define theorem:
    \newtheorem{theorem}{\bf 定理}

    % Theorem 1:
    \begin{theorem}
        % "$...$"或"/(.../)"用于数学公式插入
        $O(f) + O(g) = O(max(f, g))$
    \end{theorem}
    % Proof 1:
    \begin{proof}
        % "\left\{...\right."表示左侧大括号
        % "aligned用于多行的对齐, 其中&标记了要对齐的位置
        % "\forall"表示任意, "\exists"表示存在
        % "^..."表示上角标, "_..."表示下角标
        % "\leq"表示小于等于号, "\geq"表示大于等于号
        设
        \begin{equation}
            \left\{
            \begin{aligned}
                \nonumber
                F(N) & = O(f) \\
                G(N) & = O(g)
            \end{aligned}
            \right.
        \end{equation}
        则$\exists C_1 > 0, \exists N_1 \in \mathcal{N}$,
        使得$\forall N \geq N_1, F(N) \leq C_1f(N)$,\\
        $\exists C_2 > 0, \exists N_2 \in \mathcal{N}$,
        使得$\forall N \geq N_2, G(N) \leq C_2g(N)$.\\
        令
        \begin{equation}
            \left\{
            \begin{aligned}
                \nonumber
                C_3  & = max\{C_1, C_2\} \\
                N_3  & = max\{N_1, N_2\} \\
                h(N) & = max\{f, g\}
            \end{aligned}
            \right.
        \end{equation}
        则$\forall N > N_3$,
        $F(N) \leq C_1f(N) \leq C_1h(N) \leq C_3h(N)$.\\
        同理可得, $\forall N > N_3$,
        有$G(N) \leq C_2g(N) \leq C_2h(N) \leq C_3h(N)$.\\
        所以
        \begin{align*}
            O(f) + O(g) & = F(N) + G(N)          \\
                        & \leq C_3h(N) + C_3h(N) \\
                        & = 2C_3h(N)             \\
                        & = O(h)                 \\
                        & = O(max(f, g))
            \qedhere  % 证毕符号
        \end{align*}
    \end{proof}
    \vspace{1em}

    % Theorem 2:
    \begin{theorem}
        $O(f) + O(g) = O(f + g)$
    \end{theorem}
    % Proof 2:
    \begin{proof}
        设
        \begin{equation}
            \left\{
            \begin{aligned}
                \nonumber
                F(N) & = O(f) \\
                G(N) & = O(g)
            \end{aligned}
            \right.
        \end{equation}
        则$\exists C_1 > 0, \exists N_1 \in \mathcal{N}$,
        使得$\forall N \geq N_1, F(N) \leq C_1f(N)$,\\
        $\exists C_2 > 0, \exists N_2 \in \mathcal{N}$,
        使得$\forall N \geq N_2, G(N) \leq C_2g(N)$.\\
        令
        \begin{equation}
            \left\{
            \begin{aligned}
                \nonumber
                C_3 & = max\{C_1, C_2\} \\
                N_3 & = max\{N_1, N_2\}
            \end{aligned}
            \right.
        \end{equation}
        则$\forall N > N_3$,
        $F(N) \leq C_1f(N) \leq C_3f(N)$.\\
        同理可得, $\forall N > N_3$,
        有$G(N) \leq C_2g(N) \leq C_3g(N)$.\\
        所以
        \begin{align*}
            O(f) + O(g) & = F(N) + G(N)          \\
                        & \leq C_3f(N) + C_3g(N) \\
                        & = C_3(f(N) + g(N))     \\
                        & = O(f + g)
            \qedhere
        \end{align*}
    \end{proof}
    \vspace{1em}

    % Theorem 3:
    \begin{theorem}
        $O(f) \cdot  O(g) = O(f \cdot g)$
    \end{theorem}
    % Proof 3:
    \begin{proof}
        设
        \begin{equation}
            \left\{
            \begin{aligned}
                \nonumber
                F(N) & = O(f) \\
                G(N) & = O(g)
            \end{aligned}
            \right.
        \end{equation}
        则$\exists C_1 > 0, \exists N_1 \in \mathcal{N}$,
        使得$\forall N \geq N_1, F(N) \leq C_1f(N)$,\\
        $\exists C_2 > 0, \exists N_2 \in \mathcal{N}$,
        使得$\forall N \geq N_2, G(N) \leq C_2g(N)$.\\
        令
        \begin{equation}
            \left\{
            \begin{aligned}
                \nonumber
                C_3 & = C_1 \cdot C_2   \\
                N_3 & = max\{N_1, N_2\}
            \end{aligned}
            \right.
        \end{equation}
        则$\forall N > N_3$,
        $F(N) \leq C_1f(N)$.\\
        同理可得, $\forall N > N_3$,
        有$G(N) \leq C_2g(N)$.\\
        所以
        \begin{align*}
            O(f) \cdot O(g) & = F(N) \cdot G(N)          \\
                            & \leq C_1f(N) \cdot C_2g(N) \\
                            & = C_3(f(N) \cdot g(N))     \\
                            & = O(f \cdot g)
            \qedhere
        \end{align*}
        \qedhere
    \end{proof}
    \vspace{1em}

    % Theorem 4:
    \begin{theorem}
        $g(N) = O(f) \Rightarrow O(f) + O(g) = O(f)$
    \end{theorem}
    % Proof 4:
    \begin{proof}
        设$G(N) = O(g)$,
        则$\exists C_1 > 0, \exists N_1 \in \mathcal{N}$,
        使得$\forall N \geq N_1, G(N) \leq C_1g(N)$.\\
        所以$\forall N \geq N_1$,
        \begin{align*}
            O(f) + O(g) & = g(N) + G(N)       \\
                        & \leq g(N) + C_1g(N) \\
                        & = (C_1 + 1)g(N)     \\
                        & = O(f)
            \qedhere
        \end{align*}
        \qedhere
    \end{proof}
    \vspace{1em}

    % Theorem 5:
    \begin{theorem}
        $O(Cf) = O(f)$, C是一个常数
    \end{theorem}
    % Proof 5:
    \begin{proof}
        设$F(N) = O(f)$.\\
        则$\exists C_1 > 0, \exists N_1 \in \mathcal{N}$,
        使得$\forall N \geq N_1, F(N) \leq C_1f(N)$.\\
        同理可得, $\exists \frac{C_1}{C} > 0, \exists N_1 \in \mathcal{N}$,
        使得$\forall N \geq N_1, F(N) \leq (\frac{C_1}{C})Cf(N)$,\\
        所以, $O(Cf) = O(f)$.
        \qedhere
    \end{proof}
    \vspace{1em}

    % Theorem 6:
    \begin{theorem}
        $f = O(f)$
    \end{theorem}
    % Proof 6:
    \begin{proof}
        设$F(N) = O(f)$.\\
        则$\exists C_1 > 0, \exists N_1 \in \mathcal{N}$,
        使得$\forall N \geq N_1, F(N) \leq C_1f(N)$.\\
        所以, $O(f) = F(N) \leq C_1f(N) = f$.
        \qedhere
    \end{proof}
    \vspace{1em}

\end{sloppypar}
\end{document}
